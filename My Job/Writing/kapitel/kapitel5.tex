\chapter{Result}
Now we can use our method to determine $dS/dt$, $Pre$, $ET$ and $R$. After that, we can combine these results and discuss the trend of water storage in Ob basin and the reason for it.
\section{Equivalent water height}
As mentioned in chapter 3, We have GRACE and GRACE-FO data from 4 data centers, which are CSR, GFZ, ITSG and JPL. We can plot the total water storage anomaly from 2002 to 2019. we can see in \ref{fig:4centers} that the trends of 4 curves suit each other well. By using the methods mentioned in \ref{sec:Gaussmarkov} we can generate one timeseries with their uncertainty in this period (\ref{fig:EWHs}).
\begin{figure}[htbp]
	\centering
	\includegraphics[width=0.9\textwidth]{TWSAall} % Datei in "bilder/" bei LaTeX: eps, bei PDFLaTeX: jpg (o.ä.) 
	\caption{Equivalent water height from CSR, GFZ, ITSG and JPL between 2002 and 2020} 
	\label{fig:4centers}
\end{figure}

% figure 4 data centers label(fig:4centers)

Since Ob basin is a quite big area, it is necessary to confirm if the behavior of the total water storage are identical in the whole area, we can divide the whole area in many grids \\\\

% text needs to be finished,  label(fig:grid of ob)

From the timeseries of the TWSA we see an obvious positive trend from 2013 to 2015. Thus, the whole period can divide into 3 phases. In order to find the changing points, we first plot the $dS/dt$ \ref{fig:dsdtall} by using the equation \ref{eq:dsdt}. With the help of the matlab functions \textit{movmean} and \textit{ischange} we can see the mean value of $dS/dt$ between 2013 and 2015 are much bigger than other periods, which confirms our assumption. This big change of the trend happened between October 2012 and September 2015. Therefore, we divide the whole timeseries into following 3 periods:
\begin{itemize}
	\item Oct 2003 to Sep 2012
	\item Oct 2012 to Sep 2015
	\item Oct 2015 to Sep 2019
\end{itemize}
\begin{figure}[htbp]
	\centering
	\includegraphics[width=0.9\textwidth]{EWH} % Datei in "bilder/" bei LaTeX: eps, bei PDFLaTeX: jpg (o.ä.) 
	\caption{Equivalent water height between 2002 and 2020} 
	\label{fig:EWHs}
\end{figure}
\begin{figure}[htbp]
	\centering
	\includegraphics[width=0.9\textwidth]{dSdt} % Datei in "bilder/" bei LaTeX: eps, bei PDFLaTeX: jpg (o.ä.) 
	\caption{dS/dt between 2003 and 2020} 
	\label{fig:dsdtall}
\end{figure}
\begin{figure}[htbp]
	\centering
	\includegraphics[width=0.9\textwidth]{dSdtmovmean} % Datei in "bilder/" bei LaTeX: eps, bei PDFLaTeX: jpg (o.ä.) 
	\caption{Sudden change of dS/dt between 2003 and 2020 after using movmean} 
	\label{fig:dsdtmovmean}
\end{figure}
% dsdt figure \label(fig:dsdtall)
Then, the mean value of $dS/dt$ with uncertainty in these 3 periods can be calculated. 
\begin{figure}[htbp]
	\centering
	\includegraphics[width=0.9\textwidth]{dSdtwithnumber} % Datei in "bilder/" bei LaTeX: eps, bei PDFLaTeX: jpg (o.ä.) 
	\caption{The mean value of dS/dt in three periods} 
	\label{fig:dsdtwithnum}
\end{figure}
\clearpage
\section{Precipitation}
It was mentioned in chapter 3 that we have precipitation data from 9 resources \ref{fig:precenter}, and one timeseries with uncertainty can be generated by combining them. The spatial behavior of precipitation in this area is also interesting to us. 
\begin{figure}[htbp]
	\centering
	\includegraphics[width=0.9\textwidth]{precenter} % Datei in "bilder/" bei LaTeX: eps, bei PDFLaTeX: jpg (o.ä.) 
	\caption{Precipitation datasets} 
	\label{fig:precenter}
\end{figure}
\\
Since we have already divide the whole period into 3 phases, we are interested what we see the behavior of precipitations in this 3 phases. \ref{fig:allpre}
\begin{figure}[htbp]
	\centering
	\includegraphics[width=0.9\textwidth]{precipitationwithnumber} % Datei in "bilder/" bei LaTeX: eps, bei PDFLaTeX: jpg (o.ä.) 
	\caption{precipitation with uncertainties between 2003 and 2020 and the mean value in 3 periods} 
	\label{fig:allpre}
\end{figure}
% all precipitation in one 
\section{Evatranspiration}
Just as precipitation, we present the evatranspiration temparally ans spatially and cut the timeseries in to 3 small periods.
\begin{figure}[htbp]
	\centering
	\includegraphics[width=0.9\textwidth]{etcenter} % Datei in "bilder/" bei LaTeX: eps, bei PDFLaTeX: jpg (o.ä.) 
	\caption{Evatranspiration datasets} 
	\label{fig:etcenter}
\end{figure}
\begin{figure}[htbp]
	\centering
	\includegraphics[width=0.9\textwidth]{etall} % Datei in "bilder/" bei LaTeX: eps, bei PDFLaTeX: jpg (o.ä.) 
	\caption{evatranspiration with uncertainties between 2003 and 2020 and the mean value in 3 periods} 
	\label{fig:etall}
\end{figure}
\clearpage
\section{Runoff}
\subsection{Runoff from global datasets}
Like precipitation and evatranspiration, we have runoff data estimated from several models and meanwhile we have in-situ data till end of 2010. We can plot them in a way we did for precipitation and evatranspiration. From the \ref{fig:rcenter} we can assume that there are large differences between these models and in-situ. By calculating RMSE for these models \ref{tab:rmse} this assumption can be confirmed. The smallest RMSE for those daatsets are bigger than $8.18 \frac{mm}{month}$, which is far from what we needed.
\begin{figure}[htbp]
	\centering
	\includegraphics[width=0.9\textwidth]{runoffcenter} % Datei in "bilder/" bei LaTeX: eps, bei PDFLaTeX: jpg (o.ä.) 
	\caption{Runoff datasets} 
	\label{fig:rcenter}
\end{figure}
\begin{table}[htbp]\label{tab:rmse} \centering
	\begin{tabular}{|l|l|}
		\hline
		Datacenter  & RMSE (mm/month) \\ \hline
		ERA5        & 8.18  \\ \hline
		HTESSEL     & 10.40 \\ \hline
		LISFLOOD    & 11.92 \\ \hline
		ORCHIDEE    & 10.24 \\ \hline
		PCRGLOBWB   & 9.48  \\ \hline
		GLDAS CLSM  & 13.68 \\ \hline
		GLDAS NOAH  & 17.01 \\ \hline
		GLDAS VIC   & 25.95 \\ \hline
		SURFEX-TRIP & 22.05 \\ \hline
		W3RA        & 16.47 \\ \hline
		WaterGAP3   & 11.03 \\ \hline
	\end{tabular}
	\caption{RMSE for runoff datacenters}
\end{table}\\
Then, if we take the CDF of the difference from those models and in-situ data and set 10\% of mean in-situ runoff as quantile \ref{fig:rcdf} , it is shown that none of these datasets has achieved probability of 90\%
\begin{figure}[htbp]
	\centering
	\includegraphics[width=0.6\textwidth]{rcdf} % Datei in "bilder/" bei LaTeX: eps, bei PDFLaTeX: jpg (o.ä.) 
	\caption{CDF of differences} 
	\label{fig:rcdf}
\end{figure}\\
\subsection{Estimating runoff using quantile function and satellite altimetry}
 As mentioned in \ref{sec:runoff}, we can use the satellite to determine the water level height in Ob river basin, and we also have the in-situ data, using the methods in \ref{sec:waterlevel}, the runoff can be estimated. We have chosen several virtual station for each satellite mission. After denying the virtual stations in bad locations we get 2 water level timeseries from Envisat, 2 from Saral and 2 from Sentinel. In order to get the most accurate runoff, we choose the water level timeseries from the virtual stations, which are closer to Salekhard, where the in-situ runoff are measured. \\\\
 In \ref{fig:waterlevel} the location of those virtual stations and the water level timeseries are represented. we finally choose the blue line from Envisat, the blue line from Saral and timeseries from Sentinel-3A to generate discharge.
 \begin{figure}[htbp]
 	\centering
 	\begin{minipage}[t]{0.7\textwidth}
 		\centering
 		\includegraphics[width=0.8\textwidth]{Envisat_Ob} % Datei in "bilder/" bei LaTeX: eps, bei PDFLaTeX: jpg (o.ä.) 
 	\end{minipage}
 	\begin{minipage}[t]{0.7\textwidth}
 		\centering
 		\includegraphics[width=0.8\textwidth]{Saral_Ob} % Datei in "bilder/" bei LaTeX: eps, bei PDFLaTeX: jpg (o.ä.) 
 	\end{minipage}
 \begin{minipage}[t]{0.7\textwidth}
 	\centering
 	\includegraphics[width=0.8\textwidth]{Sentinels_Ob} % Datei in "bilder/" bei LaTeX: eps, bei PDFLaTeX: jpg (o.ä.) 
 \end{minipage}
 \caption{water level time series and the location of virtual station}
 \label{fig:waterlevel}
 \end{figure}
\\
Now we have in-situ runoff data from 2001 to 2010 and water level: